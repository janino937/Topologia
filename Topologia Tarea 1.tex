\documentclass[letter,twoside,12pt]{article}
\usepackage{amsmath}
\usepackage{amssymb}
\usepackage{amsthm}
\usepackage{fullpage}
\usepackage[spanish]{babel}
\usepackage{latexsym}
\usepackage{enumerate}
\usepackage{enumitem}
\title{Topolog\'ia I: Tarea \#1}
\newtheorem{lemma}{Lema}
\author{Jonathan Andr\'es Ni\~no Cort\'es}
\begin{document}
\maketitle
Para cada $a \in \mathbb{Z}^+$ y $b \in \mathbb{Z}$ definimos el conjunto $S(a,b) :=\{an+b:n \in \mathbb{Z}\}\subseteq \mathbb{Z}$
\begin{enumerate}[label=\textbf{\arabic*}.]
\item Pruebe que la colecci\'on $\{S(a,b):(a,b) \in \mathbb{Z^+ \times \mathbb{Z}}\}$ es base para una topolog\'ia (que llamaremos $\tau_F$) sobre $\mathbb{Z}$.
\begin{proof}
Sea $x \in \mathbb{Z}$. En primer lugar, $x \in S(1,0)$ ya que $x=1*x+0$.

En segundo lugar, sea $S(a,b)$ y $S(c,d)$ conjuntos de la colecci\'on y sea $x \in S(a,b) \cap S(c,d)$. De manera equivalente, $\underline{x=an+b=cm+d}\text{ }(*)$, para algunos $n,m \in  \mathbb{Z}$.
Se puede demostrar que $x \in S(ac,x) \subseteq S(a,b) \cap S(c,d)$.

Por un lado, $x \in S(ac,x)$, pues $x=ac*0+x$. Por otro lado, sea $y \in S(ac,x)$ (es decir, que $y=acp+x$, $p \in \mathbb{Z}$). Ahora,
\begin{eqnarray}
y&=&acp+x \nonumber
\\ &=&acp+an+b \text{ por  }(*)\nonumber
\\ &=&a(cp+n)+b\nonumber
\end{eqnarray}
Similarmente
\begin{eqnarray}
y&=&acp+x \nonumber
\\ &=&acp+cm+d \text{ por }(*)\nonumber
\\ &=&c(ap+m)+d\nonumber
\end{eqnarray}

Lo que muestra que $y \in S(a,b) \cap S(c,d)$.
\end{proof}
\item Muestra que la topolog\'ia $\tau_F$ no es la (usual) topolog\'ia discreta sobre $\mathbb{Z}$. M\'as a\'un, muestre que ning\'un conjunto finito $A \subset \mathbb{Z}$ es abierto.

\begin{proof}
Se debe exceptuar el vac\'io, que siempre es finito y abierto.

Obs\'ervese que la base esta formada por subconjuntos infinitos de $\mathbb{Z}$. Como cualquier conjunto abierto corresponde a la uni\'on de elementos de la base, se concluye que si un conjunto es abierto entonces este es infinito. 
\end{proof}
\item Muestre que $(\mathbb{Z}, \tau_{F})$ es un espacio de Hausdorff. 
\begin{proof}
Sea $x,y \in \mathbb{Z}$ tales que $x \not = y$. Sin perdida de generalidad suponga que $x < y$. Vamos a demostrar que $S(m,x)$ y $S(m,y)$, con $m =y-x+1$ son dos conjuntos abiertos que contienen a $x$ y $y$ respectivamente y que son disyuntos.

Lo primero es cierto, porque $x=0*m+x$ y $y=0*m+y$, es decir, $x \in S(m,x)$ y $y \in S(m,y)$.

Para la otra proposici\'on, supongase por contradicci\'on que no son disyuntos. Entonces tomamos $z$ en la intersecci\'on y existen $p,q \in \mathbb{Z}$ tales que $z=mp+x=mq+y$. De aqu\'i deducimos que $m(p-q)=y-x$, es decir que $m$ divide a $y-x$. Pero tenemos que $0<y-x<m$, por lo cual esto es una contradicci\'on.
\end{proof}
\item Pruebe que cada $S(a,b)$ es un conjunto cerrado (i.e. complemento de un abierto).

\begin{proof}
Veamos que el complemento de $S(a,b)$ lo podemos escribir como uni\'on de conjuntos en la base, es decir un conjunto abierto.

Vamos a demostrar que
\begin{equation}
S(a,b)^C=\bigcup_{i=1}^{a-1} S(a,b+i) \nonumber
\end{equation}

Utilizamos el m\'etodo de doble contenencia. Primero tomemos un elemento $x$ en la uni\'on. Entonces existen $n,i \in \mathbb{Z}$, tales que $0<i<a$ y $x=an+b+i$. Si suponemos por contradicci\'on que $x \in S(a,b)$, entonces $x=am+b$ para alg\'un $m \in \mathbb{Z}$.

A partir de estas dos expresiones de $x$ concluimos que 
\begin{eqnarray}
am+b&=&an+b+i\nonumber
\\ am&=&an+i\nonumber
\\ a(m-n)&=&i\nonumber
\end{eqnarray}

Es decir que $a$ divide a $i$, pero esto no es posible por el intervalo en que se encuentra $i$ $(0<i<a)$. Luego $x \in S(a,b)^C$

Para la otra contenencia si $x \not \in S(a,b)$ entonces no existe ningun $n$ tal que $x=an+b$. Por el algortimo de la divisi\'on sabemos que existen enteros $q,r \in \mathbb{Z}$ tales que $x-b=aq+r$ con $0 \leq r < q$. Pero $r \not =0$ pues de lo contrario contradicir\'ia nuestra hip\'otesis inicial. Por lo tanto, vemos que $x=aq+b+r$, es decir que $x \in S(a,b+r)$, y por lo tanto pertenece a la uni\'on.
\end{proof}

\item Muestre que
\begin{equation}
\mathbb{Z}\backslash\{-1,1\} = \bigcup_{p \text{ primo}} S(p,0) \nonumber
\end{equation}
y use los problemas 2 y 4 para concluir que existen infinitos primos.

\begin{proof}
La igualdad anterior se da como consecuencia del teorema fundamental de la aritm\'etica. Todo n\'umero entero positivo tiene una expansi\'on en primos que es \'unica. Por lo tanto, existe un $p$ primo tal que $p$ divide a $n$, es decir, $n \in (S,p)$. El n\'umero 1 (y el -1) es un caso especial ya que por convenci\'on es equivalente a la multiplicaci\'on de un conjunto vac\'io de primos, de tal manera que no hay ningun primo que divida a 1 (similarmente se cumple para el -1). Por otra parte, si $m$ es negativo entonces podemos escribirlo como $m=-1*n$ para $n>1$ y por lo tanto por transitividad de la divisi\'on existe un $p$ que divide a $m$. 

Ahora por el punto 4) tenemos que los conjuntos en la uni\'on tambi\'en son cerrados.

Por las leyes de De morgan tenemos que la uni\'on finita de conjuntos cerrados es cerrada, partiendo del hecho que la intersecci\'on finita de a biertos es abierta. Sea $\mathcal{A}$ una colecci\'on finita de conjuntos.

\begin{equation}
\bigcup_{A\in\mathcal{A}} A^C=(\bigcap_{A\in\mathcal{A}} A)^C \nonumber
\end{equation}

Vemos que la parte derecha de la igualdad es el complemento de un conjunto abierto y por lo tanto es cerrado.

Si suponemos que el n\'umero de primos es finito entonces la uni\'on es un conjunto cerrado por lo anterior y por lo tanto su complemento es abierto. Pero esto es una contradicci\'on con el punto 2) porque el conjunto $\{-1,1\}$ es finito.
\end{proof}
\end{enumerate}
\end{document}